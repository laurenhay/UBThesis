% Place abstract below.

The Standard Model \cite{} of particle physics describes the interactions and form of the known fundamental particles of our universe. It has been broadly successful in its predictions since achieving its current form in the 1970s and has seen many discoveries reinforce its predictions since. Although it can be said that the final piece of the standard model was found in 2012 with the discovery of the Higgs boson, this model does not provide a complete picture of our universe. With still open questions like the origin of matter anti-matter asymmetry, the identity of dark matter, the hierarcy of mass generations, etc... there is still room to push this model to its limits. Where we find deviation from the predictions of the standard model, we may find answers to these open questions in physics. This motivates us to measure standard model physics objects as precisely as possible to find where these deviations may live and motivate a search for new physics. Our particular focus is in intial states from particles governed by Quantum Chromodynamics (QCD), which are particularly hard to simulate due to the self-coupling of the gluon and the asymptotic freedom of the strong force. This unique behavior causes a chain reaction of decays that can be clustered into a particle jet. The better our measurements of QCD-initiated jets, the better we will be able to discern new physics from them as a background.

In particular, there is desire for quark/gluon jet taggers since some particles are expected to decay only into jets belonging to one or the other. To improve a tagger you would ideally have measurements of pure final states of each to be able to compare the performance of your tagger to data; however, quark and gluon jets are inherently hard to uncouple, so our taggers are only based on MC simulation of pure quark and gluon states. We seek to measure gluon dominant, and a mixed quark and gluon channels to help with these studies.

This dissertation presents a double differential cross-section measurement in jet mass and transverse momentum using proton-proton collisions at a center of mass energy of 13 TeV. The full legacy dataset recorded by the CMS detector in Run 2 is used, corresponding to an integrated luminosity of $137.6 \text{fb}^{-1}$. Absolute and normalized differential cross-sections are measured in two channels, with and without a jet grooming algorithm applied: inclusive dijet, and the softest jet of trijet events. Each cross section is corrected for detector effects and compared to fixed-order predictions of perturbative QCD and to simulations using various Monte Carlo event generators.
